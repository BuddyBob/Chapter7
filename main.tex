\documentclass{article}
\usepackage{amsmath,amssymb}
\usepackage{graphicx}
\usepackage[most]{tcolorbox}  
\usepackage{geometry}
\geometry{margin=1in}

\usepackage{enumitem}
\setlist{itemsep=5pt, topsep=5pt}
\usepackage{titlesec}
\titleformat{\section}{\normalfont\Large\bfseries\color{blue!70!black}}{}{0em}{}
\titleformat{\subsection}[runin]{\bfseries\color{blue!80!black}}{}{0em}{}[.]
\titleformat{\subsubsection}[runin]{\itshape\color{black}}{}{0em}{}[.]
\usepackage{xcolor}
\title{\textbf{Calculus BC Scrapbook: Chapter 7 – Differential Equations}}
\author{Thavas Antonio}
\date{May 2025}

\begin{document}

\maketitle     
\tableofcontents
\newpage

\section{Introduction}

In this unit of my scrapbook we will explore mathematical models of how systems change over time. Key topics I will cover -
modeling situations with differential equations, verifying solutions, using slope fields, applying Euler's method, solving separable equations, and modeling exponential and logistic growth. 

\medskip
\noindent
\textbf{What this scrapbook does.}  Each Big Idea (Sections 7.1–7.9) is dedicated two pages.

\begin{itemize}
  \item \textbf{Concept} – 1. definition 2. concise explanation 3. essential vocabulary 4. annotated diagrams 5. examples
  \item \textbf{Answer} – solutions if you'd like to check your answers
\end{itemize}

\vspace{10pt}

\begin{tcolorbox}[title=Chapter 7 Roadmap,colback=gray!8,colframe=cyan]
\begin{enumerate}
  \item 7.1 Modeling Situations with DEs  
  \item 7.2 Verifying Solutions for DEs  
  \item 7.3 Sketching Slope Fields  
  \item 7.4 Reasoning of Slope Fields  
  \item 7.5 Approximating Solutions with Euler’s Method 
  \item 7.6 General Solutions via Separation of Variables  
  \item 7.7 Particular Solutions with Initial Conditions  
  \item 7.8 Exponential Models with DEs  
  \item 7.9 Logistic Models with DEs
\end{enumerate}
\end{tcolorbox}
\newpage

%==================== 7.1 ====================
\section{7.1 Modeling Situations with Differential Equations}
\subsection*{Big Idea Definition}
\begin{itemize}
  \item \textbf{Differential equations} are essentially equations are the relationship between a function and its rate of change.
  \item \textbf{Order:} highest derivative present 
\end{itemize}

\subsection*{Explanation}
I think of a differential equation as \textbf{how fast something is changing at a given \textit{t}}.

\begin{itemize}
  \item \textbf{Variable} $y(t)$ = what we’re tracking (water volume, temperature, pop, etc.).
  \item \textbf{Derivative} $\dfrac{dy}{dt}$ = rate at which something changes (liters per minute, C per second, pop increase per month).
  \item A derivitive is the linking of these two: for example, $\dfrac{dy}{dt}=5$ means water is being added at a rate of 5 {unit}.
        $\dfrac{dy}{dt}=0.1y$ means population grows 10 \% every time unit.
\end{itemize}

Writing a model can be done three steps:

\begin{enumerate}
  \item \textbf{Figure out what is changing.} 
  \item \textbf{Describe the rule.} Is change constant or proportional
  \item \textbf{Translate the rule into symbols.} Your math 
\end{enumerate}

\medskip

\subsection*{Key Vocabulary}
\begin{enumerate}
    \item rate of change, derivative, independent variable, dependent variable, order, initial value problem (IVP), parameter, solution curve
\end{enumerate}

\subsection*{Worked Examples}
\begin{enumerate}
  \item \textbf{Steady Faucet (Constant Rate)}  
        A bucket is empty at $t=0$ and water flows in at \(5\text{ L/min}\).  
        \[
          \frac{dV}{dt}=5
        \]  
        In a sentence “the volume \(V\) increases 5 liters every minute.”

  \item \textbf{Rabit Pop Growth (Proportional Rate)}  
        If a rabbit population grows at 20 \% per month, where the constant 0.20 is the {growth rate}.
        \[
          \frac{dP}{dt}=0.20\,P
        \]  
        
\end{enumerate}

%==================== 7.2 ====================
\newpage
\section{7.2 Verifying Solutions for Differential Equations}

\subsection*{Big-Idea Definition}
To verify a solution means to prove that the function we came up with actually fits the DE.
We can do this by \textbf{finding the required derivative}, \textbf{sub into} the DE,
\textbf{simplify}, and make sure what is returned is \textbf{true}

\subsection*{Explanation}
Imagine Mr. Moy hands you $\dfrac{dy}{dx}=6x$ and tells that
$y=3x^{2}+7$ is the solution.  


\begin{enumerate}
  \item Differentiate Mr. Moy's $y$: \(\dfrac{dy}{dx}=6x\).
  \item Compare with the DE’s right side (also $6x$).  
  \item They match so verification works
\end{enumerate}

\subsection*{Key Vocabulary}
 explicit solution, implicit solution,
initial value

\subsection*{Worked Examples with Verifications}

\begin{enumerate}
  \item \textbf{Linear Growth}  
  Verify that \(y = 4x - 1\) solves \(\dfrac{dy}{dx} = 4\).

  \textbf{Step 1:} Find \(\dfrac{dy}{dx}\):
  \[
  \frac{dy}{dx} = 4
  \]

  \textbf{Step 2:} Compare with the right side of the DE:
  \[
  \frac{dy}{dx} = 4
  \]

  \textbf{Conclusion:} Since both sides match, \(y = 4x - 1\) is a valid solution.

  \bigskip

  \item \textbf{Exponential Growth}  
  Verify that \(y = Ce^{2x}\) solves \(\dfrac{dy}{dx} = 2y\).

  \textbf{Step 1:} Differentiate:
  \[
  \frac{dy}{dx} = 2Ce^{2x}
  \]

  \textbf{Step 2:} Compute the right-hand side:
  \[
  2y = 2(Ce^{2x}) = 2Ce^{2x}
  \]

  \textbf{Step 3:} Compare:
  \[
  \frac{dy}{dx} = 2y 
  \]

  \textbf{Conclusion:} \(y = Ce^{2x}\) is a valid solution and was proved simply through differentiating and plugging

  \bigskip

  \item \textbf{Implicit}  
  Prove that \(x^2 + y^2 = 25\) satisfies: \(x\dfrac{dy}{dx} + y = 0\)

  \textbf{Step 1:} Some solutions require implicit diff:
  \[
  \frac{d}{dx}(x^2 + y^2) = \frac{d}{dx}(25) \Rightarrow 2x + 2y\frac{dy}{dx} = 0
  \]

  \textbf{Step 2:} Simplify:
  \[
  x + y\frac{dy}{dx} = 0 \Rightarrow x\frac{dy}{dx} + y = 0
  \]

  \textbf{Conclusion:} As proven the equation given matches the DE equation
\end{enumerate}




\newpage
\section{7.3 Sketching Slope Fields}

\subsection*{Big-Idea Definition}
Simply put slope fields are very helpful for visualizing solutions to differential equations.
Additionally they are very helpful in visualizing anti derivitives
\[
\frac{dy}{dx} = f(x, y)
\]
Instead of writing equations we can model them using line segments at given points. Entering f(x,y) will give us a slope value creating cool visuals. 

\subsection*{Explanation}
Previous big ideas explain how to write equations, prove and solve them.
Slope fields are a tool to visualize the behavior of our solutions. 
\textbf{DESMOS} is a great tool to do this quickly. 

To sketch one:
\begin{enumerate}
  \item Pick a grid of \((x, y)\) points
  \item At each point, plug values into your differential equation to get the slope.
  \item Draw a small segment with that slope at each point (estimate this)
  \item Once you have a all the point slopes drawn, you may notice a flow
\end{enumerate}

\textbf{Why this matters:} This will help us understand the behaviour
\centering
\includegraphics[width=0.5\linewidth]{slopeField1.png}
\label{fig:enter-label}

\subsection*{Key Vocabulary}
slope field, direction field, tangent, solution curve

\subsection*{Worked Examples}

\begin{enumerate}
  \item \textbf{Sketch the slope field for \( \dfrac{dy}{dx} = x \).}

    \textbf{Steps:}
    \begin{itemize}
      \item Simply plug in x range [-1,1] to get slope values
      \item At \(x = 0\), the slope is 0 \(\Rightarrow\) flat line
      \item At \(x = 1\), the slope is 1 \(\Rightarrow\) 45° line.
      \item At \(x = -1\), the slope is -1 \(\Rightarrow\) lines slope downward.
    \end{itemize}


  \begin{center}
      \vspace{0.3cm}
      \includegraphics[width=0.5\textwidth]{slopeField2.png}
      \vspace{0.3cm}
  \end{center}

  \textbf{Observation:} Appears to be a bell shaped curve

  \bigskip

\item \textbf{Lets try something harder: Solve and verify that \( y = \frac{1}{x + C} \) satisfies the differential equation \( \dfrac{dy}{dx} = -y^2 \).}

\textbf{Steps:}
\begin{itemize}
  \item Lets start with the given solution: \( y = \frac{1}{x + C} \)
  \item Differentiate using the chain rule:
  \[
  \frac{dy}{dx} = \frac{d}{dx}\left(\frac{1}{x + C}\right) = (x+C)^1= -\frac{1}{(x + C)^2}
  \]
  \item Now square \( y \):
  \[
  y^2 = \left(\frac{1}{x + C}\right)^2 = \frac{1}{(x + C)^2}
  \]
  \item Compare the both. Looks like there's a match.
\end{itemize}

\vspace{0.3cm}
\newpage
\textbf{How this fits with the slope field:}
\begin{itemize}
  \item The green lines are our y function 
  \item \(\frac{dy}{dx}\) is modeled by our slope field
  \item Notice how the solution lines up perfectly with the little slope marks on the graph
\end{itemize}

\begin{center}
\vspace{0.3cm}
\includegraphics[width=0.5\textwidth]{slopeField3.png}
\vspace{0.3cm}
\end{center}

\end{enumerate}

\textbf{Conclusion:}  
The function \( y = \frac{1}{x + C} \) satisfies the differential equation \( \dfrac{dy}{dx} = -y^2 \). 


\newpage
\section{7.4 Reasoning Using Slope Fields}

\subsection{Big-Idea Definition}
Once you've drawn a slope field we need to learn how to read it. \textbf{Reasoning with slope fields} means simply looking at the direction of our segments and finding a best fit curve.

\subsection*{Explanation}
Each line segment in the slope field tells us the slope of the solution curve at that exact point. So, if we were to "drop a pencil" on the graph at any point and try to follow the little slopes, we'd trace out a solution.


\textbf{\(\Rightarrow\)} Flat segments usually mean equilibrium solutions.

\subsubsection*{Things to look for:}
\begin{itemize}
  \item Where are the slopes zero? \(\Rightarrow\) That means the curve is flat there.
  \item Where do slopes change from positive to negative or vice versa?
  \item If you’re given an initial value (like \(y(1) = 2\)), find that point and follow the slope from there.
\end{itemize}

\vspace{10pt}
\begin{figure}[h]
\caption{Sample slope field for \(\dfrac{dy}{dx} = y(1 - y)\)}
\includegraphics[width=0.6\textwidth]{slopeField4.png}
\end{figure}

\subsection*{Key Vocabulary}
equilibrium solution, stability, initial value, solution curve, direction field

\subsection*{Worked Examplees}

\begin{enumerate}
  \item \textbf{Example 1: Equilibrium Thinking}  
Let’s look at this differential equation:  
\[
\frac{dy}{dx} = y(1 - y)
\]

\textbf{\(\Rightarrow\) Step 1: When is the slope zero?}  
This means: when is \(\frac{dy}{dx} = 0\)?  
Set the right side equal to zero:
\[
y(1 - y) = 0
\Rightarrow y = 0 \text{ or } y = 1
\]

These are called \textbf{equilibrium solutions | critical points} as the slope is zero here.

\textbf{\(\Rightarrow\) Step 2: What happens above, below, and between?}
\begin{itemize}
  \item If you start between \(y = 0\) and \(y = 1\), the slope is positive \(\Rightarrow\) solutions go up.
  \item If you start above \(y = 1\), the slope is negative \(\Rightarrow\) solutions go down.
  \item If you start below \(y = 0\), the slope is positive again \(\Rightarrow\) solutions go up.
\end{itemize}

\textbf{Conclusion:}
\begin{itemize}
  \item \(y = 1\) curves move toward it.
  \item \(y = 0\) curves move away from it.
\end{itemize}

  \vspace{10pt}
  \begin{figure}[h]
  \centering
  \caption{Solutions rising or falling depending on initial value}
  \includegraphics[width=0.5\textwidth]{slopeField4.png}
  \end{figure}
  \item \textbf{Example 2: Which curve fits?}

  Suppose you're shown a slope field and three candidate solution curves—A, B, and C.

  \textbf{How to check:}
  \begin{itemize}
    \item \textbf{Find key points:} Choose some \((x, y)\) values and check the slope the curve has there.
    \item \textbf{Compare with the slope field:} Does the drawn slope match the little segment's slope?
    \item \textbf{Reject curves that cross or ignore slope directions.}
  \end{itemize}

  \begin{figure}[h]
  \centering
  \caption{Only Curve B follows the slope directions correctly. Curves A and C do not match the field}
  \includegraphics[width=0.7\textwidth]{output.png}
  \end{figure}

\item \textbf{Example 3: Predicting Long-Term Behavior}

Consider the differential equation:
\[
\frac{dy}{dx} = x - y
\]

\textbf{\(\Rightarrow\) Step 1: Where is the slope zero?}  
Set \(x - y = 0 \Rightarrow y = x\). This is a diagonal line across the plane. At every point on this line, the slope is flat (\(dy/dx = 0\)).

\textbf{\(\Rightarrow\) Step 2: Analyze what happens around it.}
\begin{itemize}
  \item If \(y < x\), then \(x - y > 0 \Rightarrow\) slope is positive \(\Rightarrow\) curve rises.
  \item If \(y > x\), then \(x - y < 0 \Rightarrow\) slope is negative \(\Rightarrow\) curve drops.
\end{itemize}


\textbf{Conclusion:}  
All solutions tend to follow the line \(y = x\) over time. Even though the actual solution may start above or below, it moves toward that line:
\[
\lim_{x \to \infty} y(x) = x
\]
\end{enumerate}

%==================== 7.5 ====================
\newpage
\section{7.5 Approximating Solutions Using Euler’s Method}

\subsection*{Big-Idea Definition}
\begin{itemize}
  \item \textbf{Euler’s Method} – Essentially a "connect the dots" or a way to estimate the solution of a DE by taking small steps along a curve's tangent lines
  \item \textbf{Simple Rule}  
        \[
          y_{\text{next}} 
          \;=\; y_{\text{current}}
          \;+\; f(x_{\text{current}},y_{\text{current}})\,\Delta x
        \]
        where \(f(x,y)=\dfrac{dy}{dx}\).
\end{itemize}

\subsection*{Explanation}
\begin{enumerate}
  \item Pick a \textbf{step size}. It will look like \(\Delta x\). The smaller value you choose the more accurate but the more tedious
  \item Start at the given \textbf{initial point} \((x_0,y_0)\).
  \item Get the slope \(f(x_0,y_0)\) and jump:\;
        \(y_1 = y_0 + f(x_0,y_0)\,\Delta x\).
  \item Move on by step size:\;
        \(x_1 = x_0 + \Delta x\).\;
        Repeat until you reach the \(x\)-value you care about.
\end{enumerate}

\medskip
\noindent
\textbf{Easy Explanation.} Think of it as driving at a constant slope for \(\Delta x\) seconds, then finding the slope at the new spot, and so on. Smaller steps = smoother ride.

\subsection*{Key Vocabulary}
step size, local error, global error, recursion, tangent line, approximation table

\subsection*{Worked Examples}

\begin{enumerate}
\item \textbf{Warm-Up}  
Estimate \(y(1)\) for 
\(\displaystyle \frac{dy}{dx}=x+y, \quad y(0)=1\)  
using \(\Delta x = 0.50\).

What this problem is asking from us is to apply Euler’s Method with a step size of \(\Delta x = 0.50\), starting at the initial point \((0,1)\), to approximate the value of \(y\) when \(x = 1\).

\begin{center}
\begin{tabular}{|c|c|c|c|}
\hline
$n$ & $x_n$ & $y_n$ & $f(x_n,y_n)=x_n+y_n$ \\ \hline
0 & 0.00 & 1.000 & 1.000 \\ \hline
1 & 0.50 & 1.500 & 2.000 \\ \hline
2 & 1.00 & 2.500 & – \\ \hline
\end{tabular}
\end{center}

\[
\boxed{\,y(1)\;\approx\;2.50\,}
\]


\item \textbf{Spicier Example}  
Use Euler’s method with \(\Delta x = 0.25\) to approximate \(y(0.5)\) for  
\(\displaystyle \frac{dy}{dx}=y-x^2, \quad y(0)=0.\)

What this problem is asking from us is to apply Euler’s Method with a step size of \(\Delta x = 0.25\), starting at the initial point \((0,0)\), to take two quarter-step jumps and approximate the value of \(y\) when \(x = 0.5\).

\begin{center}
\begin{tabular}{|c|c|c|c|}
\hline
$n$ & $x_n$ & $y_n$ & $f(x_n,y_n)=y_n-x_n^2$ \\ \hline
0 & 0.00 & 0.000 & 0.000 \\ \hline
1 & 0.25 & 0.000 & $-0.0625$ \\ \hline
2 & 0.50 & $-0.0156$ & – \\ \hline
\end{tabular}
\end{center}

\[
\boxed{\,y(0.5)\;\approx\;-0.016\;(\text{rounded})\,}
\]

\bigskip
\textit{Takeaway:} Smaller \(\Delta x\) would nudge those answers closer to the actual solution curve but you'd probably get really tired. I believe Mr. Moy doesn't mind 3-4 steps. 


\end{enumerate}

%==================== 7.6 ====================
\newpage
\section{7.6 General Solutions via Separation of Variables}

\subsection*{Big-Idea Definition}
\begin{itemize}
  \item A \textbf{separable differential equation} can be rearranged so that every \(y\)-term (and \(dy\)) lives on one side and every \(x\) term (and \(dx\)) lives on the other:
  \[
    \frac{dy}{dx}=G(x)\,H(y)
    \;\;\Longrightarrow\;\;
    \frac{1}{H(y)}\,dy = G(x)\,dx.
  \]
  \item Integrate both sides \(\displaystyle\int\) and slap on a “\(+C\)”
\end{itemize}

\subsection*{Explanation}
\begin{enumerate}
  \item \textbf{Spot the structure.}  Make sure the DE can be written as \(H(y)\,dy = G(x)\,dx\). 
  \item \textbf{Separate.}  Physically move the \(y\) stuff (including \(dy\)) to one side and the \(x\) stuff (with \(dx\)) to the other.
  \item \textbf{ \(\displaystyle\int\) Integrate both sides.}
  \item \textbf{Solve for \(y\) (if possible).} You could end up with implicit diff which makes your life harder.
\end{enumerate}

\subsection*{\centering{Key Vocabulary}}
separable DE, constant of integration, implicit solution, explicit solution, antiderivative

\subsection*{Worked Examples}

%------------------------------------------------
\begin{enumerate}
\item \textbf{Easy example}  
Find the general solution of
\[
\frac{dy}{dx}=3x^{2}\,y.
\]

\textit{Separate:}\;
\(\displaystyle\frac{1}{y}\,dy = 3x^{2}\,dx.\)

\textit{Integrate:}\;
\(\displaystyle \int\frac{1}{y}\,dy = \int 3x^{2}\,dx
\;\Rightarrow\;
\ln|y| = x^{3}+C.\)

\begin{align*}
\ln|y| &= x^{3}+C \\[6pt]
\text{Exponentiate both sides:}\qquad 
|y| &= e^{\,x^{3}+C}
      \;=\; e^{C}\,e^{\,x^{3}} \\[6pt]
\text{Lets make K absorb our }e^{C}:\qquad
|y| &= K\,e^{\,x^{3}} \\[6pt]
\text{Drop the absolute value}\pm: \;
y &= \pm K\,e^{\,x^{3}} \\[6pt]
\text{Say our }\pm K\text{can be an anything goes constant }C: \qquad
\boxed{y = C\,e^{\,x^{3}}}.
\end{align*}

%------------------------------------------------
\bigskip
\item \textbf{A Tad Trickier}  
Solve
\[
\frac{dy}{dx}=\frac{x^{2}}{1+y^{2}}.
\]

\textit{Separate:}\;
\((1+y^{2})\,dy = x^{2}\,dx.\)

\textit{Integration help along the way}
\[
\int (1 + y^{2})\,dy \;=\; \int 1\,dy \;+\; \int y^{2}\,dy
       \;=\; y \;+\; \frac{y^{3}}{3} \;+\; C_{1}.
\]

\[
\int x^{2}\,dx \;=\; \frac{x^{3}}{3} \;+\; C_{2}.
\]

\textit{Combine the two:}
\[
y + \frac{y^{3}}{3} \;=\; \frac{x^{3}}{3} \;+\; (C_{2} - C_{1})
\;\;\Longrightarrow\;\;
y + \frac{y^{3}}{3} \;=\; \frac{x^{3}}{3} \;+\; C,
\]
Once again C is our catch all constant

\textit{Implicit answer:}\;
\[
\boxed{\,y + \dfrac{y^{3}}{3} = \dfrac{x^{3}}{3} + C\,}
\]


\end{enumerate}

\noindent

%==================== 7.7 ====================
\newpage
\section{7.7 Particular Solutions with Initial Conditions}

\subsection*{Big-Idea Definition}
\begin{itemize}
    In this topic, we’ll learn about the difference between a general solution and a particular solution using separation of variables. It may be helpful to go back and review 7.6.
  \item A \textbf{particular solution} is the curve from a family that actually passes through a given point \((x_0,y_0)\).
  
  \item You find it by \textbf{plugging the initial condition} into the general solution to pin down the constant \(C\).
\end{itemize}

\subsection*{Explanation}
\begin{enumerate}
  \item \textbf{Get the general solution} could use 7.6 for this
  \item \textbf{Insert} \(x_0\) and \(y_0\) into that general solution.
  \item \textbf{Solve for \(C\)}.  
        with painful algebra you’ve got a unique \(C\).
  \item \textbf{Rewrite} the formula with that \(C\) and you have a particular solution.
\end{enumerate}

\subsection*{Key Vocabulary}
initial condition, IVP (initial value problem), particular solution, unique solution

\subsection*{Worked Examples}

%------------------------------------------------
\begin{enumerate}
\item \textbf{Easy}  
Solve the IVP
\[
\frac{dy}{dx}=3x^{2}y, \qquad y(0)=4.
\]

\textit{Step 1 – General solution (see Section 7.6):}\;
\(y = C e^{x^{3}}\).

\textit{Step 2 – Apply \(y(0)=4\):}
\[
4 = C e^{0}\quad\Longrightarrow\quad C = 4.
\]

\textit{Particular solution:}\;
\[
\boxed{y = 4\,e^{x^{3}}}.
\]

%------------------------------------------------
\bigskip
\item \textbf{Harder}  
Solve the IVP
\[
\frac{dy}{dx}=y(1-y), \qquad y(0)=\tfrac12.
\]

\textit{Step 1 – Separate and integrate:}
\[
\frac{dy}{y(1-y)} = dx
\quad\Longrightarrow\quad
\ln\!\bigl|\tfrac{y}{1-y}\bigr| = x + C.
\]

\textit{Step 2 – Exponentiate and tidy:}
\[
\frac{y}{1-y} = Ce^{\,x}.
\]

\textit{Step 3 – Apply \(y(0)=\tfrac12\):}
\[
\frac{\tfrac12}{1-\tfrac12} = C e^{0}
\;\;\Longrightarrow\;\;
1 = C.
\]

\textit{Step 4 – Solve for \(y\):}
\[
\frac{y}{1-y}=e^{\,x}
\;\;\Longrightarrow\;\;
y = \frac{e^{\,x}}{1+e^{\,x}}.
\]

\textit{Particular solution:}\;
\[
\boxed{y(x)=\dfrac{e^{\,x}}{1+e^{\,x}}}.
\]

\end{enumerate}

\noindent
\textbf{So.}  
The initial condition acts like a “GPS pin” on the slope field—only one curve goes through it, and Section 7.7 shows how not only to spot it but to name it through algebra


%==================== 7.8 ====================
\newpage
\section{7.8 Exponential Models with Differential Equations}

\subsection*{Big-Idea Definition}
\begin{itemize}
  \item When will I use math? Well this can be applied with stuff like —bacteria, money in the bank, radioactive atoms— changes so that \emph{the bigger it is, the faster it grows (or decays)}.  
        You can capture that with 
        \[
          \frac{dy}{dt}=k\,y,
        \]
        where  
        \(\;y(t)\) = “how much we have at time \(t\),”  \\
        \(k>0\) = growth constant (upward trend),  \\
        \(k<0\) = decay constant (downward trend).
  \item Solving that one-liner gives an \textbf{exponential curve}  
        \[
          y(t)=C\,e^{kt},
        \]
        where \(C=y(0)\) is simply the starting amount.  
        In plain words: keep multiplying by the same factor every equal chunk of time.
\end{itemize}

\subsection*{Explanation (Why the math makes sense)}
\begin{enumerate}
  \item \textbf{Proportionality picture.}  Every extra gram of bacteria, or dollar in your bank earns its own “mini-growth,” so the overall change stacks up proportionally.
  \item \textbf{Deriving the formula quickly.}  Separate variables:
        \(\displaystyle \frac{1}{y}\,dy = k\,dt\).  
        Integrate → \(\ln|y| = k t + C\).  
        Exponentiate → \(y = C e^{k t}\).
\end{enumerate}

\subsection*{Key Vocabulary}
exponential growth, exponential decay, growth constant

\subsection*{Worked Examples (kind of hard :(  )}
%------------------------------------------------
\begin{enumerate}
\item \textbf{Bacteria Growth – Classic Doubling}
A bacteria culture starts with \(500\) cells and doubles every 3 hours.

\begin{enumerate}[label=\alph*)]
  \item Write an equation for the population after \(t\) hours.
  \item How many bacteria are there after 10 hours?
\end{enumerate}

\textit{Step-by-step:}
\begin{itemize}
  \item Since it’s doubling, the growth is exponential.
  \item We use the model: \(\dfrac{dy}{dt} = k y\).
  \item We know it doubles in 3 hours, so use that to find \(k\):  
  \[
  2 = e^{3k} \Rightarrow k = \frac{\ln 2}{3} \approx 0.231
  \]
  \item Now plug into the model:  
  \[
  \boxed{y(t) = 500\,e^{0.231t}}
  \]
  \item After 10 hours:
  \[
  y(10) = 500\,e^{2.31} \approx 5040 \text{ cells}
  \]
\end{itemize}

\bigskip
%------------------------------------------------
\item \textbf{Carbon-14 – Decay Example}
A fossil had 10 mg of Carbon-14. It has a half-life of 5730 years.

\begin{enumerate}[label=\alph*)]
  \item Write a model for how much C-14 remains after \(t\) years.
  \item How much is left after 10,000 years?
\end{enumerate}

\textit{Step-by-step:}
\begin{itemize}
  \item This is decay, so we still use \(\dfrac{dy}{dt} = k y\), but \(k < 0\).
  \item Half-life formula gives:  
  \[
  k = \frac{-\ln 2}{5730} \approx -0.000121
  \]
  \item Plug into the formula:  
  \[
  \boxed{y(t) = 10\,e^{-0.000121t}}
  \]
  \item After 10,000 years:
  \[
  y(10000) \approx 10\,e^{-1.21} \approx 2.98 \text{ mg}
  \]
\end{itemize}

\bigskip
%------------------------------------------------
\item \textbf{Continuous Interest – Bank Account}
You put \$1200 into an account that grows at 4.5% interest, compounded continuously.

\begin{enumerate}[label=\alph*)]
  \item Write a model for the balance.
  \item How much is in the account after 8 years?
  \item When will the account reach \$3000?
\end{enumerate}

\textit{Step-by-step:}
\begin{itemize}
  \item Use the same model: \(\dfrac{dy}{dt} = 0.045 y\)
  \item So:
  \[
  \boxed{y(t) = 1200\,e^{0.045t}}
  \]
  \item After 8 years:
  \[
  y(8) \approx 1200\,e^{0.36} \approx \$1721
  \]
  \item To reach \$3000:
  \[
  3000 = 1200\,e^{0.045t} \Rightarrow t = \frac{\ln(2.5)}{0.045} \approx 18.3 \text{ years}
  \]
\end{itemize}
\end{enumerate}

\noindent
\textbf{Big Picture Reminder:}  
If something keeps growing or shrinking \emph{proportionally} to how much there is, you're looking at exponential behavior. Just find \(k\), plug it into the formula, and boom—you’ve got a time machine.





\newpage

\section{7.9 Logistic Models with Differential Equations}

\subsection*{Big-Idea Definition}
A \textbf{logistic model} is like population growth... but with common sense.

Instead of growing forever like \( \frac{dy}{dt} = ky \) - essentially rabbit population but no fox -  logistic growth includes a cap. This upper limit is called the \textbf{carrying capacity}, and it reflects things like limited food, space, or how many times Mr. Moy will let a student say “Can we go over the curve one more time?” before he says “No.”

The logistic differential equation looks like:
\[
\frac{dy}{dt} = ky\left(1 - \frac{y}{L}\right)
\]

Where:
- \( y \) = population at time \( t \)
- \( k \) = growth rate
- \( L \) = carrying capacity (the max the environment can support)

\subsection*{Explanation}
When \( y \) is small, \( \left(1 - \frac{y}{L}\right) \approx 1 \), and the growth is nearly exponential.  
But as \( y \) gets closer to \( L \), the factor \( \left(1 - \frac{y}{L}\right) \) gets smaller. The population growth slows down.  
When \( y = L \), the growth stops completely:  
\[
\frac{dy}{dt} = 0
\]

This makes logistic models much more realistic than unlimited exponential growth—because no classroom has unlimited desks, no concert has unlimited tickets, and no AP class has unlimited sanity.

\begin{figure}[h]
\includegraphics[width=0.6\textwidth]{logistic_curve_example.png}
\end{figure}

\subsection*{Key Vocabulary}
logistic growth, carrying capacity, equilibrium, saturation, inflection point, S-curve

\subsection*{Worked Example}

\textbf{Mr. Moy’s Croissant Factory }  
Suppose Mr. Moy runs a croissant bakery. He starts with 100 croissants in circulation and expects the demand to grow quickly... but there’s a max of 1000 croissants the kitchen can support (supply chains, butter shortages, etc).

He models it like this:
\[
\frac{dy}{dt} = 0.5y\left(1 - \frac{y}{1000}\right)
\]

\textbf{Step 1: What's the carrying capacity?}  
\[
L = 1000 \quad \text{(the kitchen's max croissant limit)}
\]

\textbf{Step 2: Where is growth fastest?}  
When \( y = \frac{L}{2} = 500 \) croissants. That’s when production goes beast mode.

\textbf{Step 3: Where is the slope zero?}  
\[
\frac{dy}{dt} = 0 \quad \text{when } y = 0 \text{ or } y = 1000
\]
Those are equilibrium points: zero croissants (sad) or full capacity (yay).

\textbf{Step 4: Sketch the curve.}  
Starts slow at 100, climbs fast around 500, flattens as it approaches 1000.

\begin{center}
\includegraphics[width=0.6\textwidth]{logistic_croissant_curve.png}
\end{center}
\captionof{figure}{Mr. Moy’s croissant empire in full logistic glory.}

\subsection*{Conclusion}
Logistic models are realistic because they know the world has limits.  
They’re all about fast early growth, slowing down near the top, and leveling off like the end of the AP year.

Just remember:  
\[
\frac{dy}{dt} = ky\left(1 - \frac{y}{L}\right)
\]

...and you’ll always be in the sweet spot between overgrowth and burnout. 
\end{document}